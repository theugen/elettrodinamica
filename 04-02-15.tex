\chapter*{04/02/2015}
\section*{Esercizio 3}
Nel sistema di riferimento K del laboratorio sono presenti un campo elettrico 
costante ed uniforme $\bar E = E \hat e_y$ ed un campo magnetico costante ed
uniforme $\bar B = \frac{5}{3} E \hat e_z $. Una particella di carica q e massa
m parte da ferma dall'origine all'istante t=0. Calcolare dopo quanto tempo la 
particella passa nuovamente per un punto dell'asse x.

\vspace{5 mm} % spaziatura tra il testo dell'esercizio e la risoluzione

Sono nel caso di campi $\bar E$, $\bar B$ entrambi non nulli, costanti, uniformi,
ortogonali tra loro.
Essendo B>E, posso mettermi in un sistema di riferimento K' in cui $\bar E'=0$
facendo un boost con una velocit\`a $\bar v$ giacente nel piano ortogonale
a quello identificato da $\bar E$ e $\bar B$:
\begin{equation}
\bar v = c \frac{\bar E \wedge \bar B}{B^2}
\end{equation}
\begin{equation}
\begin{cases}
E'_{\parallel} & = E_{\parallel} = 0 \\
B'_{\parallel} & = B_{\parallel} = 0 \\
E'_{\bot} & = \gamma(\bar E + \bar\beta \wedge \bar B)= \gamma(\bar E + \frac{\bar E \wedge \bar B 
\wedge \bar B}{B^2}) = \gamma(\bar E - \bar E) = 0 \\

B'_{\bot} & = \gamma(\bar B - \frac{\bar E \wedge \bar B 
\wedge \bar E}{B^2}) = \frac{\bar B}{\gamma} = (\frac{B^2-E^2}{B^2})^\frac{1}{2} \bar B\\
\end{cases}
\end{equation}

Nel sistema di riferimento in movimento ho solo campo magnetico e vedo la carica muoversi nel
piano con velocit\`a $-\bar v$ lungo x.

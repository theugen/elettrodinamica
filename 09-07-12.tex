\chapter*{09/07/2012}
\section*{Esercizio 3}
Si mostri che l'accelerazione di una particella di massa m e carica q che si muove con velocità
\textbf v sotto l'azione dei campi elettromagnetici \textbf E e \textbf B può essere scritta come

\begin{equation}
\bar a = \frac{q}{m} \sqrt{1 - \beta^2} [\bar E + \bar{\beta} \wedge \bar B - \bar \beta (\bar \beta \cdot \bar E)]
\end{equation}

Non ho informazioni sulla mutua disposizione dei campi elettromagnetici, provo
a ragionare in termini più generali.

\begin{equation}
\frac{dp^\mu}{dt} = m 
\left( \begin{array}{c}
\dot \gamma c \\
\dot \gamma \bar v + \gamma \bar{\dot v}
\end{array} \right)
 = F^\mu =
\left( \begin{array}{c}
q \bar \beta \cdot \bar E \\
q(\bar E + \bar \beta \wedge \bar B)
\end{array} \right)
\end{equation}

Conoscendo la definizione di $\gamma$ posso ricavare $\beta$ invertendo la relazione

\begin{equation}
\gamma = \frac{1}{\sqrt{1-\beta^2}} \Rightarrow \beta= \pm \frac{\sqrt{1-\gamma^2}}{\gamma}
\end{equation}

Calcolo $\dot \gamma$ dalla componente temporale del quadrivettore:

\begin{equation}
\dot \gamma = \frac{q\bar \beta \cdot \bar E}{mc}
\end{equation}

Ora, uguagliando le componenti spaziali dei quadrivettori, posso ricavare l'accelerazione:

\begin{equation}
\begin{split}
\bar{\dot v}& = \bar a = \frac{q(\bar E + \bar \beta \wedge \bar B)}{m \gamma} - \dot \gamma \bar v =
\frac{q(\bar E + \bar \beta \wedge \bar B)}{m \gamma} - \frac{q \bar \beta \cdot \bar E}{mc} 
\cdot \frac{c \sqrt(1-\gamma^2)}{\gamma} = \\
& = \frac{q(\bar E + \bar \beta \wedge \bar B)}{m \gamma} -
\frac{q\bar \beta \cdot \bar E \sqrt{1-\gamma^2}}{m\gamma^2} =
\frac{q}{m\gamma}[\bar E + \bar \beta \wedge \bar B - \bar \beta \cdot \bar E \frac{\sqrt{1-\gamma^2}}{\gamma}]=\\
& = \frac{q}{m\gamma}[\bar E + \bar \beta \wedge \bar B - \bar \beta (\bar \beta \cdot \bar E)] =  
\frac{q \sqrt{1-\beta^2}}{m}[\bar E + \bar \beta \wedge \bar B - \bar \beta (\bar \beta \cdot \bar E)] 
\end{split}
\end{equation}

dove nell'ultimo passaggio ho usato il fatto che $\gamma^{-1} = \sqrt{1-\beta^2}$.

\addcontentsline{toc}{chapter}{09/07/2012}
\addcontentsline{toc}{section}{Esercizio 3}
